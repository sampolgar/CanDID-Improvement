Each dot point contains which party is sending what elements and a short description. \\
1. Finish the scheme for master pk \& revocation/sanction/sybil resistance lists in overleaf \\
2. Work through concrete instantiation \\
3. Implement \& test
\subsection{Generating a Credential}
Credential generation involves multiple protocols and entities, split into 3 sections, 1) Porting Legacy Credentials to Pre-Credentials, 2) Pre-Credentials, and 3) Credential Generation.

\subsubsection{Legacy Credentials:} DECO \& Town Crier are 2 protocols that enable legacy credentials to be used in online credentials. DECO is a 3-party protocol between Prover $P$, Verifier $V$, and TLS Server $S$, $P$ convinces $V$ some data (possibly private to $P$) retrieved from $S$ satisfies a predicate $Pred$. DECO uses MPC to protect data and authenticity, ZKP to prove a predicate is satisfied, and distributed $V$ to decentralize the protocol.

\subsubsection{Pre-Credentials:} are used to create Credentials. A claim is a tuple $claim = \{a,v,P\}$ where $a$ is an attribute, $v$ is a value, or hiding commitment to the value e.g. $Com(v,r)$, $P$ is the provider. $PC = (claim, \pi)$ is a verifiable Pre-Credential where $\pi$ is a signature by the committee attesting to the claim satisfying. The signature signature verifies the authenticity of the pre-credential information $Pred$.\\

To generate a PC, a user $U$ picks at least $t$ committee nodes $(C_1,...,C_T)$ and executes DECO to prove their Legacy Credential $claims$ are authentic and satisfy $Pred$. $C_i$ verifies DECO proofs and generates partial $\pi_i = Sig_{sk_i}(claims)$, $U$ combines $t$ shares of $\pi_i$ to form $\pi$, the signature over the Pre-Credential that proves the attributes, values, and Provider match $Pred$ requirements. Each $PC$ can have multiple attributes and values from the same provider. \\

Depending on the master credential schema requirement, the PC, or multiple PC, is used to generate the master credential. 

\subsection{System Setup} 
Let \( a \) be the attribute used for deduplication, e.g., a government ID such as the US Social Security Number (SSN). The system is setup with credential schema with defined attributes and credential structure.
The Committee consists of \( n \) nodes \( (C_1, \ldots, C_n) \) and a Threshold Signature Scheme \( \text{TS} = (\text{TS.KGen}, \text{Sig}, \text{Comb}, \text{Vf}) \) used by the committee to issue credentials and the execute a Threshold Multi-Party Computation Pseudo Random Function $MPC_{PRF}$.

\begin{itemize}
    
    \item \textbf{IDTable} is initialized by each committee member; it will contain a tuple with the deduplication attribute/s in hidden $PRF_{Ci}(SSN)$ form and the users public key $PK_U$
    
    \item \textbf{Choose deduplication attributes} System admin selects dedup attributes $Attr = \{a_1, \ldots, a_k\}$.

    \item \textbf{Revocation List} is hosted by the system and available publicly containing $NYM_P$ or $PK_U$ and $PRF_{Ci}(name, dob)$

    \item \textbf{Sanction List} is a public list provided by governments containing the name, dob, and address of sanctioned individuals and will be available to the system.
    
    \item \textbf{ContextTable} is initialized by each committee member and stores a mapping between the
    $\{PRF_{Ci}(name, dob)$, $NYM_P$, $context\}$
    used for context-specific deduplication and revocation.
    
\end{itemize}

\subsection{User Setup} 

\begin{itemize}

    \item \textbf{Key Generation:} $(PK_U, SK_U) \leftarrow Setup(1^\lambda)$ Instantiated by the user locally to generate their identity.
    
    \item \textbf{Pseudonym Generation:} (only applicable for context credential) user generates $nonce \leftarrow \mathbb{F}$, then generates a pseudonym $NYM_P \leftarrow PRF_{SK_U}(nonce)$
and proof of correct nym generation $\pi = ZKPoK\{SK_U : NYM_P = PRF_{SK_U}(nonce)\}$

\end{itemize}

\subsection{Master Credential Issuance}

\subsubsection{Deduplication}: User blinds their deduplication attribute value and with a multi-party computation, computes a PRF representation for deduplication. \\

$U$ generates Pre-Credentials (PCs) with deduplication attribute value $v$. $U$ interacts with the committee nodes to hide $v$, in the form $\mathnormal{\hat{v}}$$= PRF(SK_C,v)$ for inserting into the IDtable $\phi$. $U$ sends $[v]$, a sharing of $v$ by committee nodes $\{C_i\}^n_{i=1}$, each node $C_i$ has $v_i$ such that $v=\Sigma_i\lambda_iv_i$ where $\lambda's$ are Lagrange coefficients 

\begin{itemize}

    \item \textit{}
    
\end{itemize}







\begin{itemize}
\item \textbf{Setup:} 
       \begin{itemize}
            \item \textit{$Attr = \{a_1, \ldots, a_k\}$. Choose deduplication attributes.}
            \item \textit{Generate credential schema with defined attributes and credential structure}
            \item \textit{Initialize a public revocation list $RL$.}
            \item \textit{Committee initializes a private IDTable $:=\phi$.}
        \end{itemize}
\end{itemize}

\begin{itemize}
\item \textbf{Key Generation:} \textit{keyGen$(1^\lambda) \rightarrow (pk, sk)$}:  Could be instantiated by the user locally for $pk_U^{master}$ or by the committee nodes in a distributed fashion $pk_C$. We assume $pk_C$ is public
\end{itemize}

\begin{itemize}
\item \textbf{Nym Generation:} \textit{nymGen$(1^\lambda, sk_U) \rightarrow (nym^{context}, nym^{context}_{verify})$}: Instantiated by the user locally to generate a pseudonym for a context credential
\end{itemize}


\begin{itemize}
\item \textbf{Pre-credential Generation:} \textit{The protocol used could be either DECO or Town Crier.}
    \begin{itemize}
    \item \textbf{For Master Credential:} \textit{issuePreCred$(pk_U^{master}, \text{Stmt}) \rightarrow \pi$}
    \item \textbf{For Context Credential:} \textit{issuePreCred$(nym^{context}, \text{Stmt}) \rightarrow \pi$}
    \end{itemize}
\end{itemize}

\begin{itemize}
    \item \textbf{issueMasterCred: \( (sk_C, sk_U^{master}, pk_U^{master}, \{({claim}_i, \pi_i)\}_{i=1}^k) \rightarrow {cred_{master}} \)}: Master credential issuance. Instantiated in a distributed fashion between the committee nodes. The credential has an identifier \( pk_U^{master} \), \( k \) input claims, and a special \textit{dedupOver} claim.
    \begin{itemize}
        \item \textbf{Deduplication}
        \begin{align*}
            \textit{let username} &= \textit{MPC}_{\text{PRF}}(\text{name, dob}) \\
            \textit{Verify username} &\notin L
        \end{align*}
    
        \item \textbf{Sanctions Screening}
        \begin{align*}
            \textit{let username} &= \textit{MPC}_{\text{PRF}}(\text{name, dob}) \\
            \pi &= \text{ZK-PoK} \{username, \, SL: username \notin SL\} \\
            \text{Verify}(SL, \pi) &= 
            \begin{cases}
                \text{"accept"} & \text{if } \pi \text{ proves that } username \notin SL \\
                \text{"reject"} & \text{otherwise}
            \end{cases}
        \end{align*}
    \end{itemize}
    
    \item \textbf{Show Master Cred (Privately/anonymously)}\\
    The user needs to prove they have a valid master credential signed by the committee. The user can prove this by generating a proof of valid signature.
    \begin{align*}
        \pi &= \textbf{NIZK-PoK}\{\sigma_{m}, C, \text{pk}_{\text{U}} : \sigma_{m} = (\text{sk}_{\text{c}}, C) \wedge C = \text{Com}(\text{pk}_{\text{U}}, r)\} \\
        \textbf{Witness} &\quad (\sigma_{m}, C, \text{pk}_{\text{U}}, r) \\
        \textbf{Statement:} &\quad \sigma_{m} \text{ is a valid signature on } C, \text{ and } C \text{ is a commitment to } \text{pk}_{\text{U}}\\
    \end{align*}
\end{itemize}


\begin{itemize}
\item \textbf{Verify Master Cred}
\begin{align*}
    \pi &= \textbf{NIZK-PoK}\{\sigma_{m}, C, \text{pk}_{\text{U}} : \sigma_{m} = (\text{sk}_{\text{c}}, C) \wedge C = \text{Com}(\text{pk}_{\text{U}}, r)\} \\
    \text{Witness:} &\quad (\sigma_{m}, C, \text{pk}_{\text{U}}, r) \\
    \text{Statement:} &\quad \sigma_{m} \text{ is a valid signature on } C, \text{ and } C \text{ is a commitment to } \text{pk}_{\text{U}} \\
    \text{Verify}(\sigma_{m}, C, \text{pk}_{\text{U}}, \pi) &= 
    \begin{cases}
        \text{"accept"} & \text{if } \pi \text{ proves that } \sigma_{m} \text{ is valid and } \ C \text{ is a commitment to } \text{pk}_{\text{U}} \\
        \text{"reject"} & \text{otherwise}
    \end{cases}
\end{align*}

\end{itemize}